%location/filename: tex/ch1.tex
%author: Anders Østevik
%Last edited: 25.09.2015
%#######--Chapter 1--#######
%Content:
%	Introduction
%	

\documentclass[main.tex]{subfiles}

\begin{document}

\chapter{Introduction}

The future upgrade of the \gls{lhc} accelerator, the \gls{slhc}, will increase the beam luminosity leading to a corresponding growth of the amount of data to be treated by the data acquisition systems. This will thus require high rate data links and high radiation tolerant \glslink{asic}{ASICs}.\\
To address these needs, the GBT architecture and transmission protocol was developed to provide the simultaneous transfer of readout data, timing and trigger signals in addition to slow control and monitoring data. \\
The GBT system can be described in two parts, where one half of the system consists of radiation hard GBT \glslink{asic}{ASICs} that will act as detectors and will thus be located in the radioactive zone. These ASICs will be used to implement bidirectional multipurpose $4.8\ \giga\bit\per\second$ optical links for the high-energy physics experiments. The other half of the system is the consists of \gls{cots} components that will through optical links receive data from the \glslink{asic}{ASICs} and 


Questions that needs answering\\
General:\\
FPGAs\\
-CycloneV\\
PLLs\\
LVDS\\
CML\\
CPRI?\\

\section{\gls{fpga}}

\Gls{fpga} is a high density \gls{ic} that is designed to be completely programmable by the customer after manufacturing. The chips are shipped completely "blank" (with the exception of \gls{fpga} evaluation boards), meaning that there are no pre-programmed logic. \Glspl{fpga} are composed of arrays of \glspl{clb} surrounded by programmable routing resources and I/O pads. \cite{weste11} 
The \gls{fpga} (described in section \ref{sec:cyclone}) used in this thesis is a re-programmable type that uses \gls{sram} to configure routing and logic functions (for a permanent program storage, the on-board \gls{flash} memory can also be used).

\subsection{Cyclone V} \label{sec:cyclone}
Cyclone V by Altera is a modern \gls{fpga} 

\section{\gls{hdl}}
The most common way to program a \gls{fpga} is by \gls{hdl}, with the major ones being SystemVerilog and \acrshort{vhdl}. \Gls{hdl} consists of text-based expressions used to describe digital hardware and use this to further simulate and synthesize the hardware described. A hardware module is simulated by applying information at the inputs and then do a check on the corresponding outputs and verify that they behave as intended. Synthesis of hardware means transforming the \gls{hdl} code into a netlist of logic and wire connections describing the hardware. \glspl{hdl} have become more and more useful as system complexity have increased. \cite{weste11} The hardware described in this thesis is done using \acrshort{vhdl}. 


\section{\Gls{transceiver} Technology}

To be able to send serial data in the gigahertz domain, a high-speed transceiver protocol is needed. The Cyclone V GT-series \glspl{fpga} supports a number of protocols that can reach speeds up to $6.144\ \giga\bit\per\second$. This section gives a general description of these protocols.

Common for all protocols described under is the fact that they treat the signals differentially. While a single ended signal involves one wire between the transmitter and receiver, with the signal swinging from a given voltage to ground, differential signals involves a pair of wires, with two signals that are identical (given equal path lengths), but with opposite polarity. This gives advantages over single ended signals such as effective isolation from power systems, minimized crosstalk and noise immunity through common-mode noise rejection. It also improves S/N ratio and effectively doubles the signal level at the output $(+v - (-v) = 2v)$, which makes it especially useful in low signal applications.\cite{douglas01}

\subsection{\gls{lvds}}

LVDS is said to be the most commonly used differential interface. The interface offers a low power consumption with a voltage swing of $350\ \milli\volt$ and good noise immunity. LVDS can deliver data rates up to $3.125\ \giga\bit\per\second$. \cite{ti08lvds}

\subsection{\gls{cml}}

For data rates that exceeds $3.125\ \giga\bit\per\second$, CML signaling is preferred. This is due to the fact that certain communication standards such as \acrshort{pcie}, \acrshort{sata} and \acrshort{hdmi}, shares consistency with CML in signal amplitude and reference to $Vcc$. CML can reach a data rate in excess of $10\ \giga\bit\per\second$, but has a higher power consumption than \gls{lvds}, with a voltage swing of approximately $800\ \milli\volt$. \cite{ti08lvds}

\subsection{CPRI}

%\gls{cpri}


\end{document}

