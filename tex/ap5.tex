%location/filename: tex/fig/ap5.tex
%author: Anders Østevik
%Last edited: 24.05.2016
%#######--Appendix - GBT Control Signals--#######
%

\documentclass[main.tex]{subfiles}

\begin{document}

\chapter{GBT Control Signals} \label{ap:gbtctrl}

The Quartus-bound \gls{issp} offers a way for the user to control and monitor the control signals of the \gls{gbt} example design. Table \ref{tab:probe} and \ref{tab:switch} gives a brief description of these control signals. Since the latency optimized version of the \gls{gbt} is not covered in this thesis, the signals related to this version is not described. The information was obtained from the \gls{gbt} \gls{fpga} video tutorials \cite{gbt_videos}. The source (S) signals are described as switches that can be manipulated by the user, like resets and pattern selects. The probe (P) signals are emulated measuring probes that monitors the different components of the GBT-FPGA.

\begin{table}[H]
\small
\begin{center}
  \begin{tabular}{| l | p{5cm} | p{8cm} |}
  \hline
    Index & Name & Description   \\
    \hline
  S00     & LOOPBACK                        & Select internal loopback inside the transceiver ('1'), or an external loopback via cabling ('0'). \\ \hline
  S01     & GENERAL RESET                     & Main reset signal of the example design. \\ \hline
  S02     & MGT TX PLL RESET                    & Individual reset signal for the MGT pll. \\ \hline
  S03     & TX RESET                        & Individual reset signal for the transceiver. \\ \hline
  S04     & RX RESET                        & Individual reset signal for the receiver. \\ \hline
  S[05..06]   & PATTERN SELECT                    & Selects the pattern that is sent through the transmitter line.
  It can send a counter value ("1h") that increments by 1, or a static value ("2h"). \\ \hline
  S07     & TX HEADER SELECTOR                  & Chooses the header of the frame: '0' for idle and '1' for data. \\ \hline
  S08     & RESET DATA \& EXTRA DATA ERROR SEEN FLAGS       & Resets \textbf{P26}.\\ \hline
  S09     & RESET RX GBT READY LOST FLAG              & Resets \textbf{P25}.\\ \hline
  S10     & TX\_FRAMECLK PHASE ALIGNER - MANUAL RESET         & Related to the latency optimized version.\\ \hline
  S[11..16] & TX\_FRAMECLK PHASE ALIGNER - GBT LINK 1 STEPS       & Related to the latency optimized version.\\ \hline
  S17     & TX\_FRAMECLK PHASE ALIGNER - ENABLE           & Related to the latency optimized version.\\ \hline
  S18     & TX\_FRAMECLK PHASE ALIGNER - TRIGGER          & Related to the latency optimized version.\\ \hline
  S[19..26]   & TX\_WORDCLOCK MONITORING - THRESHOLD UP         & Related to the latency optimized version.\\ \hline
  S[27..34]   & TX\_WORDCLOCK MONITORING - THRESHOLD LOW        & Related to the latency optimized version.\\ \hline
  S35     & TX\_WORDCLOCK MONITORING - TX RESET ENABLE      & Related to the latency optimized version.\\ \hline
  \end{tabular}
    \caption{GBT control signals overview, switches.}
  \label{tab:switch}  
\end{center}
\end{table}

\begin{table}[H]
\small
\begin{center}
  \begin{tabular}{| l | p{5cm} | p{8cm} |}
  \hline
    Index & Name & Description   \\
    \hline
  P00     & LATENCY-OPTIMIZED GBT LINK - TX   & Indicates whether the \gls{gbt} example design is using the latency optimized ('1') version or not ('0').    \\
  \hline
  P01     & LATENCY-OPTIMIZED GBT LINK - RX   & Indicates whether the \gls{gbt} example design is using the latency optimized ('1') version or not ('0').      \\
  \hline
  %S10      & Tx\_FrameCLK Phase Aligner - Manual Reset & Text \\     
  %\hline
  %S[11..16]  & Tx\_FrameCLK Phase Aligner - GBT Link 1 Steps & Each Step is 520ps \\     
  %\hline
  %S17      & Tx\_FrameCLK Phase Aligner - Enable & Enable when ...? \\    
  %\hline
  %S18      & Tx\_FrameCLK Phase Aligner - Trigger & Set to '1' for autoalignment. \\    
  %\hline 
  P02       & MGT TX PLL LOCKED               & Shows the status of the MGT pll, and remains high ('1') if the pll is locked.  \\    
  \hline
  P03     & TX\_FRAMECLK PHASE ALIGNER - PLL LOCKED     & Related to the latency optimized version. \\     
  \hline
  P04       & TX\_FRAMECLK PHASE ALIGNER - PHASE SHIFT DONE   & Related to the latency optimized version \\    
  \hline
  P[05..12]   & TX\_WORDCLOCK MONITORING - STATS        & Related to the latency optimized version \\    
  \hline
  P13     & TX\_WORDCLOCK MONITORING - TX\_WORDCLK PHASE OK & Related to the latency optimized version \\    
  \hline
  P14       & MGT READY             & Asserted high ('1') to show that the MGT transceiver is ready. \\
  \hline
  P15       & RX\_WORDCLK READY         & Asserted high ('1') to show that the RX\_WORDCLK is ready. \\
  \hline
  P16       & RX\_FRAMECLK READY        & Asserted high ('1') to show that the RX\_FRAMECLK is ready. \\
  \hline
  P17       & RX GBT READY            & Asserted high ('1') to show that the receiver of the GBT-link is ready. \\
  \hline
  P[18..23]   & RX BITSLIP NUMBER         & Related to the latency optimized version, and must remain at "00h" to indicate that the RX and TX are properly aligned. \\
  \hline
  P24       & RX HEADER IS DATA FLAG      & Indicates whether the received data has a header of the frame that is idle ('0') or data ('1'). \\
  \hline
  P25       & RX GBT READY LOST FLAG      & Indicates that the connection has been lost, and remains high until \textbf{S09} is asserted high.\\
  \hline
  P26       & RX DATA ERROR SEEN FLAGS      & Is asserted high ('1') if the pattern checker detects an indifference in the transmitted and received pattern. \\
  \hline
  P27       & RX EXTRA DATA WIDE-BUS ERROR SEEN FLAG      & Same as for \textbf{P26}, only related to the \textit{wide-bus} mode. \\  
  \hline
  P28       & RX EXTRA DATA GBT8B10B ERROR SEEN FLAG      & Same as for \textbf{P26}, only related to the \textit{8B10B} mode. \\
  \hline  
  P29     & ISSP PLL Locked             & Shows the status of the issp pll. This signal does not need to be monitored by the serial interface. \\
  \hline     
  \end{tabular}  
  \caption{GBT control signals overview, probes.}
  \label{tab:probe}
\end{center}
\end{table}

\end{document}
