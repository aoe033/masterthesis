%preample
\usepackage{graphicx} %Use this to load pictures etc.
\usepackage[usenames,dvipsnames]{xcolor} %driver-in­de­pen­dent ac­cess to sev­eral kinds of color tints, shades, tones, and mixes of ar­bi­trary col­ors.

\usepackage{float}  %Allows [H]-positioning in figures
\usepackage{amsmath} %includes align and equation. Use * t2o get rid of numbering of formulas

\usepackage[utf8]{inputenc} %Enter special characters directly (æ, ø, å)
\usepackage[T1]{fontenc} %Use a supported font (T1) that encodes the accented characters used in European languages  

%Fonts
\usepackage[scaled]{helvet} %Means of load­ing Hel­vetica, scaled so that it matches Times Ro­man (to some ex­tent).
\usepackage{courier}

%\usepackage{natbib} %Im­ple­ments both au­thor-year and num­bered ref­er­ences, as well as sup­port for other bib­li­og­ra­phy use.
\setlength{\parindent}{0.0in} %determines the length of the empty box that is added to lists by the \indent command.
\setlength{\parskip}{0.0in} %parameter that contains glue. <glue> is added between each paragraph, but not before the first or after the last paragraph.

\usepackage{fixltx2e} %Some bugfixing stuff

\usepackage{parskip} % disables auto-indentation and adds a little bit of (stretchable) space in between paragraphs

\usepackage{hyperref} %Enables hyperlinks
\usepackage{listings}%include the source code of any programming language within your document.

\usepackage{setspace} %allows more fine-grained control over line spacing. ex.: \onehalfspacing
\onehalfspacing

\usepackage[raggedright,bf,sf]{titlesec} %in­ter­face to sec­tion­ing com­mands for se­lec­tion from var­i­ous ti­tle styles. E.g., marginal ti­tles and to change the font of all head­ings with a sin­gle com­mand, also pro­vid­ing sim­ple one-step page styles. 
\usepackage{fullpage} %Sets all 4 mar­gins to be ei­ther 1 inch or 1.5 cm, and spec­i­fies the page style.

\tolerance=1000 %Parameter that tells TeX how much badness is allowable without error
%Badness is an integer from 0 to 10000 that is a measure of the quality of the spacing in any given box.

% Følgende angir bokser rundt figurer og tabeller
\floatstyle{boxed}
\restylefloat{figure}
% \restylefloat{table}
% -------------------

%Hypersetup
\hypersetup{
    %bookmarks=true,         % show bookmarks bar?
    unicode=false,          % non-Latin characters in Acrobat’s bookmarks
    pdftoolbar=true,        % show Acrobat’s toolbar?
    pdfmenubar=true,        % show Acrobat’s menu?
    pdffitwindow=false,     % window fit to page when opened
    pdfstartview={FitH},    % fits the width of the page to the window
    pdftitle={Firmware Design For Common Readout Unit},    % title
    pdfauthor={Anders Østevik},     % author
    pdfsubject={Masterthesis},   % subject of the document
    %pdfcreator={Anders Østevik},   % creator of the document
    %pdfproducer={Anders Østevik}, % producer of the document
    %pdfkeywords={keyword1, key2, key3}, % list of keywords
    pdfnewwindow=true,      % links in new PDF window
    colorlinks=false,       % false: boxed links; true: colored links
    linkcolor=red,          % color of internal links (change box color with linkbordercolor)
    citecolor=green,        % color of links to bibliography
    filecolor=magenta,      % color of file links
    urlcolor=cyan           % color of external links
}

%VHDL setup: You need the following packages:
% \usepackage{listings}
% \usepackage[usenames,dvipsnames]{xcolor}

\lstdefinelanguage{VHDL}{
  morekeywords=[1]{
    library,use,all,entity,component,if,then,case,is,port,map,in,out,end,architecture,of,
    begin,and,or,not,downto,signal,process, integer, range, to
  },
  sensitive=false,
  morekeywords=[2]{
    ieee,std_logic_1164,
    numeric_std,std_logic_arith,std_logic_unsigned,std_logic_vector,
    std_logic
  },
  morecomment=[l]{--}
}

%Define Solarized Dark Colors
% \definecolor{solGreen}{RGB}{133, 153, 0}
% \definecolor{solPurple}{RGB}{211, 54, 130}

% \definecolor{solText}{RGB}{147, 161, 161}
% \definecolor{solModifier}{RGB}{38, 139, 210}
% \definecolor{solComment}{RGB}{101, 123, 131}
% \definecolor{solSelect}{RGB}{7, 54, 66}
% \definecolor{solBak}{RGB}{0, 43, 54}

\colorlet{keyword}{blue!100!black!80} %standard
\colorlet{STD}{Lavender}
\colorlet{comment}{green!80!black!90}
\lstdefinestyle{VHDL}{
  language     = VHDL,
  basicstyle   = \footnotesize \ttfamily,
  keywordstyle = [1]\color{keyword}\bfseries,
  keywordstyle = [2]\color{STD}\bfseries,
  commentstyle = \color{comment},
  breaklines=true,                % sets automatic line breaking
  tabsize=3                                % sets default tabsize to 2 spaces
}