%location/filename: tex/ch1.tex
%author: Anders Østevik
%Last edited: 09.10.2015
%#######--Chapter 1--#######
%Content:
%	Introduction
%	

\documentclass[main.tex]{subfiles}

\begin{document}

\chapter{Introduction}
The future upgrade of the \sebo{\gls{lhc}} accelerator, the \gls{slhc}, will increase the beam luminosity leading to a corresponding growth of the amount of data to be treated by the data acquisition systems. This will thus require high rate data links and high radiation tolerant \glspl{asic}.\\

To address these needs, the \gls{gbt} architecture and transmission protocol was developed to provide the simultaneous transfer of readout data, timing and trigger signals in addition to slow control and monitoring data. \\

The \gls{gbt} system can be described in two parts: First part of the system consists of radiation hard GBT \glspl{asic} that will act as detectors and will thus be located in the radioactive zone. These \glspl{asic} are used to implement bidirectional multipurpose $4.8~\giga\bit\per\second$ optical links for the high-energy physics experiments. \\
The last part of the system is located in the counting room and consists of the \gls{cru} that will provide an interface between the detector \glspl{asic} and an online computerfarm. The \gls{cru} consists of \gls{cots} components (mainly an \acrshort{fpga}), and will through optical links receive the data from the radiation detector. \\

This thesis will mainly focus on the \gls{cru} software interface on the \acrshort{pc} side, and the physical connection between the \gls{cru} and the \gls{vldb} card, where the radiation hard \gls{asic}, the \gls{gbt}x, is located.

%\todo{General:\\PLLs\\CPRI?\\}

\section{Field Programmable Gate Array}

A \gls{fpga} is a high density \gls{ic} that is designed to be completely programmable by the customer after manufacturing (i.e. when the chip is shipped and "in the field"). The chips are shipped completely "blank", meaning that there are no pre-programmed logic.\footnote{With the exception of \gls{fpga} evaluation boards, which comes shipped with a pre-programmed hardware setup for demo purposes.} An \gls{fpga} can either be re-programmed using \gls{sram} technology, or one-time programmed by burning antifuses. The latter method makes it less prone to soft errors when exposed to radiation.\\

\Glspl{fpga} are composed of arrays of \glspl{clb} surrounded by programmable routing resources and I/O pads. A \gls{clb} consists of a \gls{lut} together with a clock and simple write logic, and uses the address bus of the \gls{lut} as the function input pins and the value at the selected address as the function output. \glspl{lut} are considered fast logic, since computing a complex function only requires a single memory lookup. The \gls{lut} can be made out of an \gls{sram} memory-block \cite{weste11}. \\

The \gls{fpga} (section \ref{sec:cyclone}) used in this thesis is a re-programmable type. It stores the user hardware setup in an \gls{sram} memory to configure routing and logic functions (for a permanent program storage, the \gls{fpga} can be configures to store the hardware setup in the on-board \gls{flash} memory, which then programs the \gls{sram} when powered on).


\subsection{Altera's Cyclone V GT} \label{sec:cyclone}

The \gls{fpga} used in this thesis is the Cyclone V GT by Altera. GT indicates that the \gls{fpga} has transceivers that supports speeds up to $6~\giga\bit\per\second$ \cite{altera_cvoverview15}. It was chosen for this thesis mainly because of the on-board transceivers that are capable of reaching speeds that surpass the requirements of the \gls{gbt}-\gls{fpga} \gls{mgt}, i.e $4.8~\giga\bit\per\second$. Originally, a Terasic Cyclone V GX development board was intended to be used for this thesis. The Terasic board has advantages over the Cyclone V GT board in terms of communication with the outside world such as on-board Usb-to-Uart (more on this in chapter \ref{chap:sercom}. However, it was discovered that the transceivers on the Terasic board were not fast enough for the \gls{gbt} \gls{mgt}; maximum supported transceiver speed is only $3.125~\giga\bit\per\second$ \cite{altera_cvoverview15}. Because of this, the more powerful Cyclone V GT \gls{fpga} development board was ordered from the Altera web-pages, replacing the Terasic.


\section{Hardware Description Language}

The most common way to program an \gls{fpga} is by \gls{hdl}, with the major ones being SystemVerilog and \acrshort{vhdl}. \Gls{hdl} consists of text-based expressions used to describe digital hardware and use this to further simulate and synthesize the hardware described. A hardware module is simulated by applying information at the inputs and then do a check on the corresponding outputs and verify that they behave as intended. Synthesis of hardware means transforming the \gls{hdl} code into a netlist of logic and wire connections describing the hardware. This can then be compiled into an \gls{fpga} which re-wires the available internal \glspl{clb} according to the given instructions. \glspl{hdl} have become more and more useful as system complexity have increased \cite{weste11}. The hardware described in this thesis is done using \acrshort{vhdl}, a strongly typed \gls{hdl} that is known for its capability of describing parallel processes.\\ 

Below is a small example describing the logic of a D Flip-Flop using \acrshort{vhdl}:

\lstset { %
    language=VHDL,
    backgroundcolor=\color{black!5}, % set backgroundcolor
    basicstyle=\footnotesize,% basic font setting
}

\begin{lstlisting}[frame=single] 
-- Example of a D Flip-Flop triggered on the rising edge of the clock.
library IEEE; 
use IEEE.std_logic_1164.all; 
 
entity DFF_high is 
  port( 
    D : in STD_LOGIC; 
    Q : out STD_LOGIC; 
    Qbar : out STD_LOGIC; 
    RESET : in STD_LOGIC; 
    CLK : in STD_LOGIC 
  ); 
end DFF_high; 
 
architecture rtl of DFF_high is 
begin 
 
process(CLK,RESET) 
begin 
  if (RESET='1') then 
    Q <= '0'; 
    Qbar <= '1'; 
  elsif (CLK'event and CLK='1') then -- Can also use rising_edge(clk)
    Q <= D; 
    Qbar <= not D; 
  end if; 
end process;
  
end rtl; 
\end{lstlisting}


\section{Cyclone V Transceiver Technology}

To be able to send serial data in the gigahertz domain, a high-speed transceiver is required. The Cyclone V GT-series \glspl{fpga} supports a number of transceiver technologies through the \gls{hsmc} physical interface that can reach speeds up to $6.144\ \giga\bit\per\second$. This section gives a general description of some of these protocols.

\subsection{Differential Signals} \label{subsec:diffsig}

Common for all protocols described is the fact that the signals are treated differentially. 

While a single ended signal involves one conductor between the transmitter and receiver, with the signal swinging from a given voltage to ground; differential signals involves a conductor pair with two signals that are identical, but with opposite polarity. The pair would ideally have equal path lenghts in order to have zero return currents, avoiding problems like \textit{EMI}. In addition, placing the signals as close as possible to one another will give benefits in terms of common noise rejection \cite{douglas01}.\\

When done correctly, differential signals have advantages over single ended signals such as effective isolation from power systems, minimized crosstalk and noise immunity through common-mode noise rejection. It also improves S/N ratio and effectively doubles the signal level at the output $(+v - (-v) = 2v)$, which makes it especially useful in low signal applications. The disadvantage comes in an increase in pin count and space required, since differential signals consists of two wires instead of one \cite{douglas01}.

\subsection{Low-Voltage Differential Signaling}

\gls{lvds} is said to be the most commonly used differential interface. The interface offers a low power consumption with a voltage swing of $350\ \milli\volt$ and good noise immunity. LVDS can deliver data rates up to $3.125\ \giga\bit\per\second$ \cite{ti08lvds}.\\

The Cyclone V GT board has 17 \gls{lvds} channels available on the \gls{hsmc} port A connector. The channels have the ability to transmit and receive data at a rate up to $840\ \mega\bit\per\second$, with support for serialization and deserialization through internal logic. \cite{altera_cvoverview15}

\subsection{Current-Mode Logic}

For data rates that exceeds $3.125\ \giga\bit\per\second$, \gls{cml} signaling is preferred. This is due to the fact that certain communication standards such as \acrshort{pcie}, \acrshort{sata} and \acrshort{hdmi}, shares consistency with CML in signal amplitude and reference to $Vcc$. CML can reach a data rate in excess of $10\ \giga\bit\per\second$, but has a higher power consumption than \gls{lvds}, with a voltage swing of approximately $800\ \milli\volt$ \cite{ti08lvds}.\\

The Cyclone V GT board has 4 Pseudo-\gls{cml} (PCML) channels available on both port A and B \gls{hsmc} connectors. The channels have the ability to transmit and receive data at a rate up to $5.0\ \giga\bit\per\second$, just over the $4.8\ \giga\bit\per\second$ range required by the \gls{gbt} \gls{mgt}. \cite{altera_cyclonekit}


%\subsection{CPRI}

%\subsection{Phase-Locked Loops}

%A \gls{pll} is a device



\chapter{The Gigabit Transceiver} \label{chap:gbt}

\todo{History, mgt, Short about the electronic components, gbt-sca, gbtx}

\section{Encoding modes}

The "GBT-Frame" mode, which is Reed-Solomon Based; the "8b10b" mode; and the "Wide-Bus" mode which is without encoding.

\section{GBT-FPGA Core}
The following sections describe the different components that makes up the GBT-FPGA Core. The information was obtained by reading the GBT-FPGA User Guide \cite{gbt_fpga}.

\subsection{GBT Bank}
The GBT Bank is defined as the top module of the \gls{gbt}-\gls{fpga} Core. It integrates up to four GBT Links and contains the ports required to operate the GBT Links.

\subsection{GBT Link}
The \gls{gbt} Link is the actual channel of the link. It is composed of three components: GBT Tx, GBT Rx, and the \gls{mgt}. The following subsections gives a brief description of these components.

\subsubsection{GBT Tx}

The \gls{gbt} Tx component is responsible for scrambling and encoding data before transmitting it through the \gls{mgt}.

\subsubsection{GBT Rx}

The \gls{gbt} Rx component is responsible for receiving, decoding and de-scrambling the data through the \gls{mgt}.

\subsubsection{Multi-Gigabit Transceiver}
The \gls{mgt} is responsible for the transmitting, receiving, serialization and de-serialization of the \gls{gbt} data. It is divided into a transmitter and a receiver part.\\ The transmitter contains a \gls{piso} with two input clocks; one for parallel data and one for serial data. It shifts in $40~\bit$ words from the GBT Tx with a reference clock of $120~\mega\hertz$, serializes the data and sends it out with the help of a dedicated Tx \gls{pll} that generates a serial clock of $2400 \mega\hertz$.\\ The receiver contains a \gls{cdr} block, a \gls{sipo}, a RXRECCLK Phase Aligner block and a Barrel-shifter. 

%\gls{cpri}


%\begin{align}
%	 b_1 &= s_{11}a_1 + s_{12}a_2 \\
%	 s_{21}a_1 + s_{22}a_2 &= b_2
%\end{align}


\end{document}

