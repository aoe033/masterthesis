%location/filename: tex/fig/serial_gui.tex
%author: Anders Østevik
%Last edited: 09.05.2016
%#######--PC to CRU serial interface--#######
%Content:
%   Interface/Communications illustrating figure between \gls{pc} and FPGA
%   

\documentclass[main.tex]{subfiles}

\usetikzlibrary{shapes,arrows}
\newcommand{\mx}[1]{\mathbf{\bm{#1}}} % Matrix command
\newcommand{\vc}[1]{\mathbf{\bm{#1}}} % Vector command

\begin{document}

% We need layers to draw the block diagram
\pgfdeclarelayer{background}
\pgfdeclarelayer{foreground}
\pgfsetlayers{background,main,foreground}

% Define a few styles and constants
\tikzstyle{PCs}=[draw, fill=blue!20, text width=5em, 
    text centered, minimum height=2.5em]
\tikzstyle{ann} = [above, text width=5em]
\tikzstyle{fpgas} = [PCs, text width=5em, fill=black!10, rounded corners, dashed, 
    minimum height=10em, minimum width=15em]
\def\blockdist{6.3}
\def\btmdist{-0.7}
\def\topdist{1.0}
\def\edgedist{2.5}

\begin{tikzpicture}
    \node (fpga) [fpgas] {};
    % Note the use of \path instead of \node at ... below. 
    \path (fpga)+(-\blockdist,\topdist) node (interface) [PCs] {Interface};
    \path (fpga)+(-\blockdist,\btmdist) node (com) [PCs] {COM port};

    \path (fpga)+(-1.8,\btmdist) node (uart) [PCs] {UART};
    \path (fpga)+(-1.8,\topdist) node (dec) [PCs] {UART decoder};

    \path (fpga)+(1.8,\topdist) node (top) [PCs] {UART top};
    \path (fpga)+(1.8,\btmdist) node (gbt) [PCs] {GBT example};
    
    % Unfortunately we cant use the convenient \path (fromnode) -- (tonode) 
    % syntax here. This is because TikZ draws the path from the node centers
    % and clip the path at the node boundaries. We want horizontal lines, but
    % the PCs and fpga blocks aren't aligned horizontally. Instead we use
    % the line intersection syntax |- to calculate the correct coordinate
    \path [draw, <->] (interface.south) -- node [right] {} 
        (com.north -| interface.south) ;
    % We could simply have written (interface) .. (fpga.140). However, it's
    % best to avoid hard coding coordinates
    \path [draw, ->] (com.10) -- node [above] {$tx$} 
        (uart.west |- com.10);
    \path [draw, <-] (com.-10) -- node [above] {$rx$} 
        (uart.west |- com.-10);

    \path [draw, <->] (uart.north) -- node [right] {} 
        (dec.south -| uart.north) ;

    \path [draw, <->] (dec.east) -- node [right] {} 
        (dec.east -| top.west) ;

    \path [draw, <->] (gbt.north) -- node [right] {} 
        (top.south -| gbt.north) ;        

    %\node (PC) [above of=interface] {PC};
    \node (PC) [below of=com] {PC};

    \path (fpga.south east)+(-3.2,0.4) node (FPGA) {FPGA};

    %\node (FPGA) [below of=uart] {FPGA};

    %\path (fpga.south west)+(1.2,-0.4) node (INS) {PC to CRU serial interface};
    
    % Now it's time to draw the colored IMU and INS rectangles.
    % To draw them behind the blocks we use pgf layers. This way we  
    % can use the above block coordinates to place the backgrounds   
    \begin{pgfonlayer}{background}
        % Compute a few helper coordinates
        %\path (interface.west |- fpga.north)+(-0.5,0.3) node (a) {};
        %\path (INS.south -| fpga.east)+(+0.3,-0.2) node (b) {};
        %\path[fill=yellow!20,rounded corners, draw=black!50, dashed]
            %(a) rectangle (b);
        \path (interface.north west)+(-0.2,0.2) node (a) {};
        %\path (PC.north -| interface.west)+(-0.2,0.2) node (a) {heia};
        \path (PC.south -| interface.east)+(+0.2,-0.2) node (b) {};
        \path[fill=blue!10,rounded corners, draw=black, dashed]
            (a) rectangle (b);
    \end{pgfonlayer}
\end{tikzpicture}


\end{document}