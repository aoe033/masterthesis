%%%%%%%%%%%%%%%%%%%%%%%%%%%%%%%%%%%%%%%%%
% Beamer Presentation
% LaTeX Template
% Version 1.0 (10/11/12)
%
% This template has been downloaded from:
% http://www.LaTeXTemplates.com
%
% License:
% CC BY-NC-SA 3.0 (http://creativecommons.org/licenses/by-nc-sa/3.0/)
%
%%%%%%%%%%%%%%%%%%%%%%%%%%%%%%%%%%%%%%%%%

%----------------------------------------------------------------------------------------
%	PACKAGES AND THEMES
%----------------------------------------------------------------------------------------

\documentclass[aspectratio=43]{beamer}

\mode<presentation> {

% The Beamer class comes with a number of default slide themes
% which change the colors and layouts of slides. Below this is a list
% of all the themes, uncomment each in turn to see what they look like.

%\usetheme{default}
%\usetheme{AnnArbor}
%\usetheme{Antibes}
%\usetheme{Bergen}
%\usetheme{Berkeley}
%\usetheme{Berlin}
%\usetheme{Boadilla} %!
%\usetheme{CambridgeUS}
%\usetheme{Copenhagen}
%\usetheme{Darmstadt} %!
%\usetheme{Dresden}
%\usetheme{Frankfurt}
%\usetheme{Goettingen}
%\usetheme{Hannover}
%\usetheme{Ilmenau}
%\usetheme{JuanLesPins}
%\usetheme{Luebeck}
%\usetheme{Madrid}
%\usetheme{Malmoe}
%\usetheme{Marburg}
%\usetheme{Montpellier}
%\usetheme{PaloAlto}
%\usetheme{Pittsburgh}
%\usetheme{Rochester}
\usetheme{Singapore}
%\usetheme{Szeged}
%\usetheme{Warsaw}

% As well as themes, the Beamer class has a number of color themes
% for any slide theme. Uncomment each of these in turn to see how it
% changes the colors of your current slide theme.

%\usecolortheme{albatross}
%\usecolortheme{beaver}
%\usecolortheme{beetle}
%\usecolortheme{crane}
%\usecolortheme{dolphin}
%\usecolortheme{dove}
%\usecolortheme{fly}
%\usecolortheme{lily}
%\usecolortheme{orchid}
%\usecolortheme{rose}
%\usecolortheme{seagull}
%\usecolortheme{seahorse}
%\usecolortheme{whale}
%\usecolortheme{wolverine}

%\setbeamertemplate{footline} % To remove the footer line in all slides uncomment this line
\setbeamertemplate{footline}[page number] % To replace the footer line in all slides with a simple slide count uncomment this line

\setbeamertemplate{navigation symbols}{} % To remove the navigation symbols from the bottom of all slides uncomment this line
}

\usefonttheme{default}
\usepackage{graphicx} % Allows including images
\usepackage{booktabs} % Allows the use of \toprule, \midrule and \bottomrule in tables
\usepackage[utf8]{inputenc}
\usepackage{geometry}
\usepackage{tikz}
\usetikzlibrary{positioning}
\usepackage[absolute,overlay]{textpos}
\usepackage[mediumspace,squaren,binary]{SIunits}


% beamer: How to place images behind text (z-order)
% (http://tex.stackexchange.com/a/134311)
\makeatletter
\newbox\@backgroundblock
\newenvironment{backgroundblock}[2]{%
  \global\setbox\@backgroundblock=\vbox\bgroup%
    \unvbox\@backgroundblock%
    \vbox to0pt\bgroup\vskip#2\hbox to0pt\bgroup\hskip#1\relax%
}{\egroup\egroup\egroup}
\addtobeamertemplate{background}{\box\@backgroundblock}{}
\makeatother

%----------------------------------------------------------------------------------------
%	TITLE PAGE
%----------------------------------------------------------------------------------------


\defbeamertemplate*{title page}{customized}[1][]
{
\center
\inserttitlegraphic\par
\usebeamerfont{institute}\insertinstitute\par
%\medskip
\noindent\rule{10cm}{0.8pt}
\usebeamerfont{title}\usebeamercolor[fg]{title page}\inserttitle\par
\noindent\rule{10cm}{0.8pt}
  %\usebeamerfont{subtitle}\usebeamercolor[fg]{subtitle}\insertsubtitle\par
  
  \usebeamerfont{author}\usebeamercolor[fg]{normal text}\bigskip\insertauthor\par
  \bigskip
  \usebeamerfont{date}\insertdate\par
}

 \title[CRU Interface Design]{\textbf{Interface Design for the \\ Gigabit Transceiver Common Readout Unit}} % The short title appears at the bottom of every slide, the full title is only on the title page
\titlegraphic{\includegraphics[width=.2\textwidth]{../img/uib-emblem-svart}}

 \author{Anders Østevik} % Your name
 \institute[Department of Physics and Technology] % Your institution as it will appear on the bottom of every slide, may be shorthand to save space
 {
 Department of Physics and Technology \\
 %University of Bergen \\ % Your institution for the title page
 \medskip
 \textmd{Master Thesis} % Your email address
 }
 \date{June 2016} % Date, can be changed to a custom date

%------------------------------------------------

\begin{document}

\setbeamercolor{title page}{fg=black}
\setbeamercolor{frametitle}{fg=blue}
\setbeamercolor{normal text}{fg=black}

\setbeamerfont*{title}{family=\sffamily,series=\bfseries,size=\large}
\setbeamerfont*{author}{family=\sffamily,series=\mdseries, size=\scriptsize}

%\setbeamerfont*{caption}{size=\footnotesize}

\makeatletter
\newenvironment{noheadline}{
    \setbeamertemplate{headline}{}
    \addtobeamertemplate{frametitle}{\vspace*{-0.9\baselineskip}}{}
}{}
\makeatother

\begin{noheadline}
\begin{frame}
\titlepage % Print the title page as the first slide
\end{frame}
\end{noheadline}

%{\usebackgroundtemplate{% 
%\tikz[overlay,remember picture] \node[opacity=0.1] at ([xshift=-2.5cm,yshift=2.4cm]current page.south east) { \includegraphics[
%width=0.4\dimexpr\paperwidth-2.5cm\relax]{../img/uib-emblem-svart}}; }

{\usebackgroundtemplate{% 
\tikz[overlay,remember picture] \node[opacity=0.06] at ([xshift=-5.2cm,yshift=-3.5cm]current page.center) { \includegraphics[
width=0.2\dimexpr\paperwidth-2.5cm\relax]{../img/uib-emblem-svart}}; }

\begin{frame}
\frametitle{Overview} % Table of contents slide, comment this block out to remove it
\tableofcontents % Throughout your presentation, if you choose to use \section{} and \subsection{} commands, these will automatically be printed on this slide as an overview of your presentation
\end{frame}

%----------------------------------------------------------------------------------------
%	PRESENTATION SLIDES
%----------------------------------------------------------------------------------------

%------------------------------------------------
\section{Introduction} % Sections can be created in order to organize your presentation into discrete blocks, all sections and subsections are automatically printed in the table of contents as an overview of the talk
\subsection{LHC Upgrade} 
\subsection{Gigabit Transceiver System}
\subsection{Primary Objectives}
%------------------------------------------------ 

\begin{frame}
\frametitle{LHC Upgrade}

\begin{backgroundblock}{7cm}{2.8cm}
\includegraphics[width=5cm]{../img/lhc.jpg}
\end{backgroundblock}

\begin{itemize}
\item Large Hadron Collider (LHC)
	\begin{itemize}
	\item Particle accelerator
	\item $27~\kilo\meter$ circular tunnel
	\item 13 TeV
	\end{itemize}
\item High-Lumiosity LHC
	\begin{itemize}
	\item 10x beam lumiosity
	\item Increase in radiation \\ and amount of data
	\item $\rightarrow$ Gigabit Transceiver
	\end{itemize}
\end{itemize}

\end{frame}

%------------------------------------------------

\begin{frame}
\frametitle{Gigabit Transceiver System}

\begin{backgroundblock}{7cm}{2.8cm}
\includegraphics[width=5cm]{../img/gbtsys}
\end{backgroundblock}

\begin{itemize}
\item On-detector - Custom ASICs
	\begin{itemize}
	\item GBTx, GBT-SCA, VTTx/VTRx
	\item E-links
	\end{itemize}
\item Off-detector - Control room
	\begin{itemize}
	\item CRU (FPGA)
	\item \textgreater~ $4.8~\giga\bit\per\second$ transceivers
	\item GBT-FPGA
	\end{itemize}
\item Optical communication
	\begin{itemize}
	\item Timing and Trigger Control (TTC)
	\item Data Acquisition	(DAQ)
	\item Slow Control (SC)
	\end{itemize}
\end{itemize}

\end{frame}

%------------------------------------------------

\begin{frame}
\frametitle{Gigabit Transceiver System}

\begin{backgroundblock}{1.5cm}{7cm}
\includegraphics[width=10cm]{../img/gbtframe}
\end{backgroundblock}

\begin{backgroundblock}{7cm}{2.8cm}
\includegraphics[width=5cm]{../img/gbtsys}
\end{backgroundblock}

\begin{itemize}
\item Encoding modes
	\begin{itemize}
	\item GBT-Frame
	\item 8B/10B
	\item Wide-Bus
	\end{itemize}
\end{itemize}

\end{frame}

%------------------------------------------------

\begin{frame}
\frametitle{GBT-FPGA}

%\begin{figure}
\begin{backgroundblock}{7.5cm}{2.8cm}
\includegraphics[width=5cm]{../img/gbtex}
\end{backgroundblock}
%\caption{GBT example design \cite{gbt_fpga}}
%\end{figure}

\begin{itemize}
\item Firmware library for \\ Altera/Xilinx FPGAs
\item GBT Link
	\begin{itemize}
	\item "Standard", "Latency-Optimized"
	\item GBT Rx, GBT Tx, GBT MGT 
	\end{itemize}
\item GBT-example Design
\end{itemize}

\end{frame}

%------------------------------------------------

\begin{frame}
\frametitle{Versatile Link Demo Board}
%\begin{figure}
\begin{backgroundblock}{2.5cm}{3cm}
\includegraphics[width=0.6\paperwidth]{../img/vldbex}
\end{backgroundblock}
%\caption{GBT example design \cite{gbt_fpga}}
%\end{figure}
\end{frame}

%------------------------------------------------

\begin{frame}
\frametitle{Primary Objective}

\begin{itemize}
\item Design a CRU control interface software
	\begin{itemize}
	\item Serial communication between PC and CRU
	\end{itemize}
\item Design a HSMC-to-VLDB PCB
	\begin{itemize}
	\item Connection between CRU and VLDB 
	\end{itemize}
\end{itemize}
\end{frame}

%------------------------------------------------

%------------------------------------------------
\section{PCB Design}
%------------------------------------------------
\subsection{Design Discussion} 
\subsection{High-Speed PCB Design} 
%------------------------------------------------

\begin{frame}
\frametitle{PCB Design}

\begin{backgroundblock}{6.5cm}{6.5cm}
\includegraphics[width=0.5\paperwidth]{../img/HSMC_52_53_hsmcspec.pdf}
\end{backgroundblock}

\begin{itemize}
\item 
\end{itemize}

\end{frame}

\begin{frame}
\frametitle{PCB Design}

\begin{backgroundblock}{1.3cm}{4cm}
\includegraphics[width=0.8\paperwidth]{../img/pcbdia.pdf}
\end{backgroundblock}

\end{frame}

%------------------------------------------------

\begin{frame}
\frametitle{Blocks of Highlighted Text}
\begin{block}{Block 1}
Lorem ipsum dolor sit amet, consectetur adipiscing elit. Integer lectus nisl, ultricies in feugiat rutrum, porttitor sit amet augue. Aliquam ut tortor mauris. Sed volutpat ante purus, quis accumsan dolor.
\end{block}

\begin{block}{Block 2}
Pellentesque sed tellus purus. Class aptent taciti sociosqu ad litora torquent per conubia nostra, per inceptos himenaeos. Vestibulum quis magna at risus dictum tempor eu vitae velit.
\end{block}

\begin{block}{Block 3}
Suspendisse tincidunt sagittis gravida. Curabitur condimentum, enim sed venenatis rutrum, ipsum neque consectetur orci, sed blandit justo nisi ac lacus.
\end{block}
\end{frame}

%------------------------------------------------
\begin{frame}
\frametitle{Multiple Columns}
\begin{columns}[c] % The "c" option specifies centered vertical alignment while the "t" option is used for top vertical alignment

\column{.45\textwidth} % Left column and width
\textbf{Heading}
\begin{enumerate}
\item Statement
\item Explanation
\item Example
\end{enumerate}

\column{.5\textwidth} % Right column and width
Lorem ipsum dolor sit amet, consectetur adipiscing elit. Integer lectus nisl, ultricies in feugiat rutrum, porttitor sit amet augue. Aliquam ut tortor mauris. Sed volutpat ante purus, quis accumsan dolor.

\end{columns}
\end{frame}

%------------------------------------------------
\section{PCB Design}
%------------------------------------------------
\subsection{LHC Upgrade}

%------------------------------------------------

\begin{frame}
\frametitle{Table}
\begin{table}
\begin{tabular}{l l l}
\toprule
\textbf{Treatments} & \textbf{Response 1} & \textbf{Response 2}\\
\midrule
Treatment 1 & 0.0003262 & 0.562 \\
Treatment 2 & 0.0015681 & 0.910 \\
Treatment 3 & 0.0009271 & 0.296 \\
\bottomrule
\end{tabular}
\caption{Table caption}
\end{table}
\end{frame}

%------------------------------------------------

\begin{frame}
\frametitle{Theorem}
\begin{theorem}[Mass--energy equivalence]
$E = mc^2$
\end{theorem}
\end{frame}

%------------------------------------------------

\begin{frame}[fragile] % Need to use the fragile option when verbatim is used in the slide
\frametitle{Verbatim}
\begin{example}[Theorem Slide Code]
\begin{verbatim}
\begin{frame}
\frametitle{Theorem}
\begin{theorem}[Mass--energy equivalence]
$E = mc^2$
\end{theorem}
\end{frame}\end{verbatim}
\end{example}
\end{frame}

%------------------------------------------------

\begin{frame}
\frametitle{Figure}
Uncomment the code on this slide to include your own image from the same directory as the template .TeX file.
%\begin{figure}
%\includegraphics[width=0.8\linewidth]{test}
%\end{figure}
\end{frame}

%------------------------------------------------

\begin{frame}[fragile] % Need to use the fragile option when verbatim is used in the slide
\frametitle{Citation}
An example of the \verb|\cite| command to cite within the presentation:\\~

This statement requires citation \cite{p1}.
\end{frame}

%------------------------------------------------

\begin{frame}
\frametitle{References}
\footnotesize{
\begin{thebibliography}{99} % Beamer does not support BibTeX so references must be inserted manually as below
\bibitem[Smith, 2012]{p1} John Smith (2012)
\newblock Title of the publication
\newblock \emph{Journal Name} 12(3), 45 -- 678.
\end{thebibliography}
}
\end{frame}

%------------------------------------------------

\begin{frame}
\Huge{\centerline{Thank you!}}
\end{frame}

%----------------------------------------------------------------------------------------

\end{document} 