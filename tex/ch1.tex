%location/filename: tex/ch1.tex
%author: Anders Østevik
%Last edited: 09.10.2015
%#######--Chapter 1--#######
%Content:
%	Introduction
%	

\documentclass[main.tex]{subfiles}

\begin{document}

\chapter{Introduction}

The future upgrade of the \gls{lhc} accelerator, the \gls{slhc}, will increase the beam luminosity leading to a corresponding growth of the amount of data to be treated by the data acquisition systems. This will thus require high rate data links and high radiation tolerant \glspl{asic}.\\
To address these needs, the \gls{gbt} architecture and transmission protocol was developed to provide the simultaneous transfer of readout data, timing and trigger signals in addition to slow control and monitoring data. \\ \\
The \gls{gbt} system can be described in two parts, where one half of the system consists of radiation hard GBT \glspl{asic} that will act as detectors and will thus be located in the radioactive zone. These \glspl{asic} will be used to implement bidirectional multipurpose $4.8\ \giga\bit\per\second$ optical links for the high-energy physics experiments. \\
The other half of the system is located in the counting room and consists of a \gls{cru} that will provide an interface between the detector \glspl{asic} and an online computerfarm. The \gls{cru} consists of \gls{cots} components and will through optical links receive data from the radiation detector. \\
This master thesis will focus on development of control and data flow firmware, development of signal processing firmware and the software interface to a PC.


Questions that needs answering\\
General:\\
FPGAs\\
-CycloneV\\
PLLs\\
LVDS\\
CML\\
CPRI?\\

\section{\gls{fpga}}

\Gls{fpga} is a high density \gls{ic} that is designed to be completely programmable by the customer after manufacturing (i.e. when the chip is shipped and "in the field"). The chips are shipped completely "blank" (with the exception of \gls{fpga} evaluation boards), meaning that there are no pre-programmed logic. \Glspl{fpga} are composed of arrays of \glspl{clb} surrounded by programmable routing resources and I/O pads. A \gls{clb} consists of a \gls{lut} together with a clock and simple write logic, and uses the address of the \gls{lut} as the function input and the value at the selected address as the function output. \glspl{lut} are considered fast logic, since computing a complex function only requires a single memory lookup. The \gls{lut} can be made out of a \gls{sram}-block. \cite{weste11} 
The \gls{fpga} (section \ref{sec:cyclone}) used in this thesis is a re-programmable type that stores its "hardware" in a \gls{sram} to configure routing and logic functions (for a permanent program storage, the on-board \gls{flash} memory can also be used).

\subsection{Cyclone V} \label{sec:cyclone}
Cyclone V by Altera is a modern \gls{fpga} 

\section{\gls{hdl}}
The most common way to program a \gls{fpga} is by \gls{hdl}, with the major ones being SystemVerilog and \acrshort{vhdl}. \Gls{hdl} consists of text-based expressions used to describe digital hardware and use this to further simulate and synthesize the hardware described. A hardware module is simulated by applying information at the inputs and then do a check on the corresponding outputs and verify that they behave as intended. Synthesis of hardware means transforming the \gls{hdl} code into a netlist of logic and wire connections describing the hardware. \glspl{hdl} have become more and more useful as system complexity have increased. \cite{weste11} The hardware described in this thesis is done using \acrshort{vhdl}, a strongly typed \gls{hdl} that is capable of describing parallel processes. 


\section{\Gls{transceiver} Technology}

To be able to send serial data in the gigahertz domain, a high-speed transceiver protocol is needed. The Cyclone V GT-series \glspl{fpga} supports a number of protocols that can reach speeds up to $6.144\ \giga\bit\per\second$. This section gives a general description of these protocols.

\subsection{Differential Signals} \label{subsec:diffsig}

Common for all protocols described under is the fact that they treat the signals differentially. 

While a single ended signal involves one conductor between the transmitter and receiver, with the signal swinging from a given voltage to ground, differential signals involves a conductor pair, with two signals that are identical, but with opposite polarity. The pair would ideally have equal path lenghts in order to have zero return currents, avoiding problems like \textit{EMI}. In addition, placing the signals as close as possible to one another will also give benefits in terms of common noise rejection.\cite{douglas01}

When done correctly, differential signals have advantages over single ended signals such as effective isolation from power systems, minimized crosstalk and noise immunity through common-mode noise rejection. It also improves S/N ratio and effectively doubles the signal level at the output $(+v - (-v) = 2v)$, which makes it especially useful in low signal applications. The dissadvantage comes in an increase in pin count and space required, since differential signals consists of two wires instead of one.

\subsection{\gls{lvds}}

LVDS is said to be the most commonly used differential interface. The interface offers a low power consumption with a voltage swing of $350\ \milli\volt$ and good noise immunity. LVDS can deliver data rates up to $3.125\ \giga\bit\per\second$. \cite{ti08lvds}

\subsection{\gls{cml}}

For data rates that exceeds $3.125\ \giga\bit\per\second$, CML signaling is preferred. This is due to the fact that certain communication standards such as \acrshort{pcie}, \acrshort{sata} and \acrshort{hdmi}, shares consistency with CML in signal amplitude and reference to $Vcc$. CML can reach a data rate in excess of $10\ \giga\bit\per\second$, but has a higher power consumption than \gls{lvds}, with a voltage swing of approximately $800\ \milli\volt$. \cite{ti08lvds}

\subsection{CPRI}

%\gls{cpri}


%\begin{align}
%	 b_1 &= s_{11}a_1 + s_{12}a_2 \\
%	 s_{21}a_1 + s_{22}a_2 &= b_2
%\end{align}


\end{document}

